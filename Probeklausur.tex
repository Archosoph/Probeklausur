% !TEX TS-program = pdflatex
% !TEX encoding = UTF-8 Unicode

\documentclass[11pt]{report} % use larger type; default would be 10pt

\usepackage[utf8]{inputenc} % set input encoding (not needed with XeLaTeX)

%%% PACKAGES GENERAL

\usepackage{lmodern}
\usepackage[T1]{fontenc}
\usepackage{esvect}
\usepackage{mathrsfs}
\usepackage[left=2.5cm, right=2.5cm, tmargin=3.0cm, bottom=2.0cm]{geometry}
\usepackage{amssymb}
\usepackage{enumerate}
\usepackage[inline]{enumitem}
\usepackage{german}
\usepackage{titlesec}
\usepackage{amsmath}
\usepackage{MnSymbol}
\usepackage{makeidx}
\usepackage{amsthm}
\usepackage{soul}
\usepackage{dsfont}
\usepackage{mathtools}
\usepackage{tabularx}
\usepackage{array} % for better arrays (eg matrices) in maths
\usepackage{subfig} % make it possible to include more than one captioned figure/table in a single float

%%% HEADERS & FOOTERS
\usepackage{fancyhdr} % This should be set AFTER setting up the page geometry
\pagestyle{fancy} % options: empty , plain , fancy
\renewcommand{\headrulewidth}{0pt} % customise the layout...
\lhead{}\chead{}\rhead{}
\lfoot{}\cfoot{\thepage}\rfoot{}

%%% SECTION TITLE APPEARANCE
\usepackage{sectsty}
\allsectionsfont{\sffamily\mdseries\upshape} % (See the fntguide.pdf for font help)
% (This matches ConTeXt defaults)

\newcommand*{\defeq}{\mathrel{\vcenter{\baselineskip0.5ex \lineskiplimit0pt
                     \hbox{\scriptsize.}\hbox{\scriptsize.}}}%
                     =}

%%% NEW COMMANDS


% Enumitem Defines
\SetLabelAlign{CenterWithParen}{\hfil(\makebox[1.0em]{#1})\hfil}
\SetLabelAlign{Center}{\hfil(#1)\hfil}

%% Paired delimiters using mathtools
% declaring the norm operator. Leaving the argument blank automatically enters a \cdot
\DeclarePairedDelimiterX\norm[1]\lVert\rVert{
	\ifblank{#1}{\:\cdot\:}{#1}
}

% declaring the abs operator. Leaving the argument blank automatically enters a \cdot
\DeclarePairedDelimiterX\abs[1]\lvert\rvert{
	\ifblank{#1}{\:\cdot\:}{#1}
}

% just to make sure it exists
\providecommand\given{}
% can be useful to refer to this outside \Set
\newcommand\SetSymbol[1][]{%
	\nonscript\:#1\vert
	\allowbreak
	\nonscript\:
	\mathopen{}
}
\DeclarePairedDelimiterX\set[1]\{\}{%
\renewcommand\given{\SetSymbol[\delimsize]}
#1
}

\newcommand\bR{\mathbb{R}}
\newcommand\bC{\mathbb{C}}
\newcommand\bQ{\mathbb{Q}}
\newcommand\bN{\mathbb{N}}
\newcommand\bK{\mathbb{K}}
\newcommand\bZ{\mathbb{Z}}

\newcommand\unit{\mathds{1}}

\newcommand\im{\mathsf{im}}
\newcommand\ggT{\mathsf{ggT}}

%%% END Article customizations
\begin{document}

	\begin{center}
	{\Large Lineare Algebra I}\\
	{\large (Inoffizielle) Probeklausur}\\
	\medskip
	{\sc Tutoren der Linearen Algebra I}\\ 
	\end{center}

    \noindent    
	Zur Vorlesung von Prof. Dr. Fabien Morel, Dr. Andrei Lavrenov und Oliver Hendrichs im Wintersemester 2024-25 \\
	\begin{center}
        \Large 
        \textbf{BITTE LEST DEN GANZEN TEXT} 
	\end{center}

	
	\medskip\noindent
	\underline{Bearbeitungsvorschlag}:\\
		Wir raten euch dringend ein Klausurszenario nachzustellen und alle Aufgaben eigenständig und ohne Hilfsmittel über einen Zeitraum von 120 Minuten zu bearbeiten.\\
		Falls ihr in diesen 120 Minuten nicht fertig werdet, dann empfehlen wir euch, eure Lösung dennoch abzugeben - oder den Rest außerhalb dieser Zeit zu bearbeiten\footnote{und euch für eine stimmige Selbsteinschätzung zu merken, welchen Teil ihr außerhalb der regulären Zeit gelöst habt und somit in einer echten Klausur nicht geschafft hättet.}.\\
		Tipps und Hinweise zur Bearbeitung findet ihr auch unter https://antonzakrewski.github.io/ - auf dieser Seite werden wir auch das pdf der Klausur hochladen.
	
	\medskip\noindent
	\underline{Abgabe}:\\
		Eure Lösung kann digital über eure campus.lmu Mailadresse an alle Tutoren bis zum 03.01.2025 per E-Mail an die folgende Adresse verschickt werden: la1-probeklausur@web.de\\
        Gebt eure Lösung bis zum 03.01.2025 als PDF-Datei benannt nach Schema vorname\textunderscore nachname.pdf ab\footnote{sehr gerne auch mit \LaTeX.}. \\
		Wir werden versuchen, diese Probeklausur so zu korrigieren, wie wir auch die echte Klausur korrigieren würden. Gebt eure Lösungen also stets mit Lösungsweg an. Orientiert euch bei der Ausführlichkeit eurer Lösungen an der Vorlesung. Ihr dürft Aussagen aus der Vorlesung, den Übungsblättern und den Tutoriumsblättern verwenden, müsst das aber an dieser Stelle kennzeichnen.\\
        Wenn ihr möchtet, dass diese Abgabe von einer bestimmten Person korrigiert wird, schreibt das sowohl auf die Abgabe als auch in den Betreff der Mail. Wir werden versuchen, diesen Wünschen, soweit möglich, nachzukommen. Es bietet sich an, den eigenen Tutor zu wählen, damit dieser eventuell im Tutorium personalisiertes Feedback geben kann. \\
        Falls wir mehr Abgaben erhalten, als erwartet, werden wir nur ausgewählte Aufgaben korrigieren. In jedem Fall werden wir fristgerechte Abgaben bevorzugt korrigieren.\\

	\medskip\noindent
	\underline{Form}:\\
		Da wir das in unserer Freizeit machen, behalten wir uns vor unsaubere Abgaben nicht zu korrigieren, damit wir wirklich mit korrigieren und nicht mit entziffern beschäftigt sind. 
        Eine saubere Abgabe beinhaltet:
        \begin{itemize}[noitemsep]
	        \item Eure E-Mail Adresse im PDF
	        \item Lesbare Schrift
	        \item Nummerierte Abgaben
	        \item Keine Schmierzettel
	        \item Dunkle Schrift auf hellem Hintergrund
        \end{itemize}


	
	\medskip\noindent
	\underline{Disclaimer}:\\
		Das ist keine offizielle Probeklausur, insofern sind alle Angaben ohne Gewähr und die (komplett freiwillige) Abgabe, sowie Korrektur haben keinerlei Einfluss auf das Vorlesungs- oder Übungsmodul zur linearen Algebra. Die echte Klausur wird nicht von uns erstellt und keiner von uns hat die echte Klausur bisher gesehen. Somit können wir auch keine Aussage darüber treffen, wie ähnlich diese Aufgaben zu den Klausuraufgaben sind. Wir haben uns an Aufgaben aus Altklausuren der letzten Jahre orientiert.\\
		Die Weitergabe der Probeklausur ist erlaubt und erwünscht. Wenn ihr irgendjemanden kennt, der daran Interesse haben könnte, dann leitet sie also gern weiter.\\
		Da das Semester noch nicht zu Ende ist und somit noch nicht der gesamte Stoff behandelt wurde, deckt diese Probeklausur natürlich den Teil der Vorlesung, der im neuen Jahr drankommt, nicht ab. 
	
	\pagebreak\noindent
	{\bf Aufgabe 1}\,(6 Punkte)\\
		Sei $\bK$ ein Körper, sei $V$ ein $\bK$-Vektorraum, seien $U_1, U_2 \leq V$ Unterräume und sei $U = U_1 + U_2$. Zeige, dass die folgenden Aussagen äquivalent sind:
		\begin{enumerate}
			\item 
				Seien $u_1 \in U_1$ und $u_2 \in U_2$ sodass $u_1 + u_2 = 0$ gilt, dann ist $u_1 = u_2 = 0$.
				
			\item 
				Für jedes $u \in U$ ist die Darstellung $u = u_1 + u_2$ für $u_1 \in U_1, u_2 \in U_2$ eindeutig.
				
			\item 
				$U_1 \cap U_2 = \set{0}$.
		\end{enumerate}
		
	\medskip\noindent
	{\bf Aufgabe 2}\,(2+2+3 Punkte)\\
		Sei $(G, \ast)$ eine Gruppe, $a \in G$. Wir definieren die Abbildungen $f_a: G \rightarrow G, f_a(x) \coloneq x \ast a$ und $g: G \rightarrow G, g(x) \coloneq x^{-1}$.
		\begin{enumerate}
			\item 
				Zeige: $f_a$ und $g$ sind bijektive Abbildungen.
				
			\item 
				Ist $f_a$ bzw. $g$ ein Homomorphismus?
		
			\item 
				Wir definieren
				\begin{equation*}
					\mathcal{G} = \set{f_a \given a \in G}.
				\end{equation*}
				Zeige, dass $\mathcal{G}$ mit der Komposition von Abbildungen eine Gruppe bildet.
		\end{enumerate}
		
	\medskip\noindent
	{\bf Aufgabe 3}\,(4 Punkte)\\
		Zeige, dass die $n$-ten  komplexen Einheitswurzeln
		\begin{equation*}
			\set{w_n^k \given k = 0, \ldots, n-1}
		\end{equation*}
		mit $w_n = \exp(2\pi i / n)$ eine abelsche Gruppe bezüglich der Multiplikation bilden.
		
	\medskip\noindent
	{\bf Aufgabe 4}\,(1 + 2 + 1 Punkte)\\
		Es seien
		\begin{equation*}
			v_1 = \begin{pmatrix} 1 \\ 2 \\ 3 \end{pmatrix}, \quad v_2 = \begin{pmatrix} 2 \\ 2 \\ 6 \end{pmatrix}, \quad v_3 = \begin{pmatrix} 0 \\ 1 \\ 0 \end{pmatrix}, \quad v_4 = \begin{pmatrix} 0 \\ 17 \\ 42 \end{pmatrix} \in \bR^3
		\end{equation*}
		gegeben.
		\begin{enumerate}
			\item 
				Sind $v_1, v_2, v_3, v_4$ linear unabhängig?
				
			\item 
				Bestimme $<v_1, v_2, v_3, v_4>$.
				
			\item 
				Gib eine Basis für den Vektorraum $<v_1, v_2, v_3, v_4>$ an.
		\end{enumerate}
				
	\medskip\noindent
	{\bf Aufgabe 5}\,(1 + 3 + 2 Punkte)\\
		Es seien
		\begin{equation*}
			w_1 = \begin{pmatrix} 1 \\ 1 \\ 0 \end{pmatrix}, \quad w_2 = \begin{pmatrix} 1 \\ 2 \\ 3 \end{pmatrix}, \quad w_3 = \begin{pmatrix} 0 \\ 1 \\ 2 \end{pmatrix} \in \bR^3
		\end{equation*}
		gegeben.
		\begin{enumerate}
			\item 
				Zeige, dass $w_1, w_2, w_3$ eine Basis von $\bR^3$ bilden. 
				
			\item 
				Bestimme die beiden Basiswechselmatrizen zwischen der Standardbasis $e_1, e_2, e_3$ und $w_1, w_2, w_3$.
				
			\item 
				Sei
				\begin{align*}
					\varphi: \bR^3 &\rightarrow \bR^3\\
					(x_1, x_2, x_3) &\mapsto (x_1 + 3x_3, x_2 + x_3, 7x_1)
				\end{align*}
				gegeben. Bestimme die Darstellungsmatrix von $\varphi$ bezüglich der Basen $(w_1, w_2, w_3)$ nach $(e_1, e_2, e_3)$.
		\end{enumerate}
		
	\medskip\noindent
	{\bf Aufgabe 6}\,(10 Punkte)\\
		Untenstehend sind 10 mathematische Aussagen. Kreuze an, ob diese wahr oder falsch sind (Eine Aussage ist wahr, wenn sie immer gilt. Eine Aussage ist falsch, wenn es mindestens ein Gegenbeispiel gibt.), die Antworten müssen nicht begründet werden.\\
		Jede richtige Antwort gibt einen Punkt.\\
		
		\renewcommand{\arraystretch}{1.5}
		\medskip\noindent
		\begin{tabularx}{\textwidth}{l X l l}
			& & \text{Wahr} & \text{Falsch} \\
			\hline
			(i) & Jedes Erzeugendensystem eines Vektorraums ist linear unabhängig. & $\bigcirc$ & $\bigcirc$ \\
			(ii) & Ist $X$ eine linear unabhängige Teilmenge eines Vektorraums, so ist jede Teilmenge von $X$ wieder linear unabhängig. & $\bigcirc$ & $\bigcirc$ \\
			(iii) & Ist $V$ ein endlich erzeugter Vektorraum und ist $U \subseteq V$ ein Untervektorraum mit $\dim(U) = \dim(V)$, so ist $U = V$. & $\bigcirc$ & $\bigcirc$ \\
			(iv) & Die Vektoren $(1,0,1), (0,1,1), (1,1,0)$ bilde eine $\bR$-Basis von $\bR^3$. & $\bigcirc$ & $\bigcirc$ \\
			(v) & Es gibt einen Untervektorraum von $\bR^3$, der genau zwei Vektoren enthält. & $\bigcirc$ & $\bigcirc$ \\
			(vi) & Sind $f: X \rightarrow Y$ und $g: Y \rightarrow Z$ Abbildungen zwischen nichtleeren Mengen, und ist $g \circ f$ surjektiv, so ist $f$ surjektiv. & $\bigcirc$ & $\bigcirc$ \\
			(vii) & Sind $f: X \rightarrow Y$ und $g: Y \rightarrow Z$ Abbildungen zwischen nichtleeren Mengen und ist $g \circ f$ injektiv, so ist $f$ injektiv. & $\bigcirc$ & $\bigcirc$ \\
			(viii) & Ist $f: X \rightarrow Y$ eine Abbildung zwischen nichtleeren Mengen, und ist $U \subseteq X$, so ist $f^{-1}(f(U)) = U$. & $\bigcirc$ & $\bigcirc$ \\
			(ix) & Ist $f: X \rightarrow Y$ eine Abbildung zwischen nichtleeren Mengen, und ist $V \subseteq Y$, so ist $f(f^{-1}(V)) = V$. & $\bigcirc$ & $\bigcirc$ \\
			(x) & Es gibt mindestens einen $\bZ$-Vektorraum. & $\bigcirc$ & $\bigcirc$ \\
		\end{tabularx}			
	
\end{document}
